%*************************************************
% In this file the abstract is typeset.
% Make changes accordingly.
%*************************************************

\addcontentsline{toc}{section}{چکیده}
\newgeometry{left=2.5cm,right=3cm,top=3cm,bottom=2.5cm,includehead=false,headsep=1cm,footnotesep=.5cm}
\setcounter{page}{1}
\thispagestyle{empty}

~\vfill

\subsection*{چکیده}
\begin{small}
\baselineskip=0.7cm

در سال‌های اخیر با توجه به افزایش چشم گیر استفاده از شبکه های کامپیوتری و نیازمندی این شبکه ها به دینامیک بالا به منظور اعمال تغییرات و برنامه ریزی سریع، مفهوم نسبتا جدیدی به نام شبکه های تعریف شده بر مبنای نرم افزار \LTRfootnote{Software Defined Network (SDN)} پدید آمده است. این شبکه ها با نگاهی مجدد به طراحی تجهیزات شبکه و جداسازی لایه کنترلی \LTRfootnote{Control Plane} از لایه هدایت داده \LTRfootnote{Data Plane} هر تجهیز باعث ایجاد امکان مدیر مرکزی، یکپارچه سازی و جداسازی بخش تصمیم گیرنده از پیچیدگی های بخش فیزیکی شده است.\\
در معماری سه لایه ای این شبکه ها ارتباط بخش کنترلی با بخش هدایت داده از اهمیت بالایی برخوردار است. پروتکل استاندارد اوپن فلو \LTRfootnote{Openflow} یکی مهم ترین پروتکل های ارتباطی بین لایه کنترل و لایه هدایت داده است که در حال حاظر به صورت وسیعی در عمل و همچنین در تحقیقات مورد استفاده قرار گرفته است. در این پروژه برآن شدیم که بررسی دقیقی درباره این پروتکل مهم داشته باشیم و همچنین ویژگی های نسخه جدید آن را بررسی کنیم.\\

\noindent\textbf{کلمات کلیدی:} شبکه های نرم افزار محور، پروتکل اوپن فلو، لایه کنترل و تصمیم گیر ، لایه هدایت داده. \\
\noindent\textbf{کلمات کلیدی انگلیسی:} 
\\\lr{Software Defined Network, SDN, Openflow, Control Plane, Data Plane, Southbound Protocol}

\end{small}